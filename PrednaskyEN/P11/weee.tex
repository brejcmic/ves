%Beamer class
\documentclass{beamer}

\usepackage[czech]{babel}
\usepackage[utf8]{inputenc}
\usepackage{fontenc}
\usepackage{tgheros}
\usepackage{array}
\usepackage{color}
\usepackage{hyperref}

\usetheme{Antibes}
\usecolortheme{crane}


\title[BE1M13VES]{BE1M13VES}
\subtitle[Manufacturing of Electrical Components] {Manufacturing of Electrical Components}
\author[Brejcha]{Michal Brejcha}
\institute[CTU]{CTU in Prague}
\date[Prague, 2017]{Prague, 2017}

\newtheorem{myDef}{}

\begin{document}
%------------------------------------------------------------------------------
%Uvodni slajd
%------------------------------------------------------------------------------
\frame{\titlepage}

\begin{frame}
\frametitle{Overview} 
\tableofcontents
\end{frame}

\AtBeginSection[]
{
  \begin{frame}
    \frametitle{TOPIC}
    \tableofcontents[currentsection]
  \end{frame}
}

%------------------------------------------------------------------------------
%EU Market
%------------------------------------------------------------------------------
\section{\texorpdfstring{EU Market}{EU Market}}
%------------------------------------------------------------------------------
	\begin{frame}
    \frametitle{EU Members}
		\begin{center}
			\begin{tabular}{c l}
			\begin{tabular}{c}\includegraphics[width=0.6\linewidth]{obr01_EUmapa.png}\end{tabular} & \tiny \begin{tabular}{l l}
			\textbf{EU members:}& \\
			Austria,& Italy,\\
			Belgium,& Latvia,\\
			Bulgaria,& Lithuania,\\
			Croatia,& Luxembourg,\\
			Cyprus,& Malta,\\
			Czech Republic,& Netherlands,\\
			Denmark,& Poland,\\
			Estonia,& Portugal,\\
			Finland,& Romania,\\
			France,& Slovakia,\\
			Germany,& Slovenia,\\
			Greece,& Spain,\\
			Hungary,& Sweden,\\
			Ireland,& United Kingdom\\
		\end{tabular}
		
		\end{tabular}
		\end{center}
	\end{frame}
%------------------------------------------------------------------------------
	\begin{frame}
    \frametitle{Documents}
		\small
		\begin{center}
			\begin{tabular}{m{0.4\linewidth} m{0.5\linewidth}}
			\includegraphics[scale=0.25]{obr02_dokumenty.png} &
			
			\begin{itemize}
				\item EU Directives are obligatory for all EU members,
				\item each country includes the directives in its legal system,
				\item the act of parliament (laws) and government regulations fulfill this task in the legal system of Czech Republic,
				\item the standards are considered as recommendation (not obligatory) in Czech Republic,
			\end{itemize}
			\end{tabular}
		\end{center}
	\end{frame}
%------------------------------------------------------------------------------
	\begin{frame}
    \frametitle{Documents}
		\small
		\begin{center}
			\begin{tabular}{m{0.4\linewidth} m{0.5\linewidth}}
			\includegraphics[scale=0.25]{obr02_dokumenty.png} &
			
			\begin{itemize}
				\item \textcolor{blue}{meeting standards requirements is an assumption for meeting Directive requirements},
				\item a product fulfilling the directive requirements is marked by CE symbol and the declaration of conformity is released.
			\end{itemize}
			\end{tabular}
		\end{center}
	\end{frame}
%------------------------------------------------------------------------------
	\begin{frame}
    \frametitle{CE Mark}
		\textbf{The symbol is defined in regulation ES 765/2008, which is complementary to Decision No. 768/2008/EC on a common framework for the marketing of products:}
		\begin{center}
			\includegraphics[scale=0.45]{obr03_ZnShodyCE.png} 
		\end{center}
	\end{frame}
%------------------------------------------------------------------------------	
	\begin{frame}
    \frametitle{Declaration of Conformity Content - Decision 768/2008/EC}
		\small
		\begin{enumerate}
			\item A product identification number (serial number).
			\item The name and address of the manufacturer or his authorized representative;
			\item \textbf{This declaration of conformity is issued under the sole responsibility of the manufacturer (or installer).}
			\item Object of the declaration (identification of product allowing traceability. It may include a photograph, where
appropriate);
			\item \textbf{The object of the declaration described above is in conformity with the relevant Community harmonization
legislation:} List of directives
		\end{enumerate}

	\end{frame}
%------------------------------------------------------------------------------	
	\begin{frame}
    \frametitle{Declaration of Conformity Content - Decision 768/2008/EC}
		\small
		\begin{enumerate}
			\setcounter{enumi}{5}
			\item References to the relevant harmonized standards used or references to the specifications in relation to which
conformity is declared.
			\item Where applicable, the notified body... (name, number)... performed... (description of intervention)... and issued the certificate.
			\item Additional information.
		\end{enumerate}
		
		\begin{flushright}
		Signed for and on behalf of: ............................
		\end{flushright}
	\end{frame}
%------------------------------------------------------------------------------	
	\begin{frame}
    \frametitle{Declaration of Conformity Example}
		
		\begin{center}
				\includegraphics[scale=0.35]{obr04_prohlaseniES.png}
		\end{center}
	\end{frame}
%------------------------------------------------------------------------------	
	\begin{frame}
    \frametitle{Notes}
		
		\begin{itemize}
			\item Products fulfilling requirements of EU directives are free to move across EU market.
			\item Test results from notified laboratory (notified body) are acknowledged in all member countries.
			\item Producer is responsible for his products. He must elaborate technical documentation and record complaints.
		\end{itemize}
	\end{frame}
%------------------------------------------------------------------------------
%Motivation for Solving WEEE
%------------------------------------------------------------------------------
\section{\texorpdfstring{Motivation for Solving WEEE}{Motivation for Solving WEEE}}
%------------------------------------------------------------------------------
	\begin{frame}
    \frametitle{Problems of Electronic and Electrical Waste}
		
		\textcolor{blue}{\textbf{WEEE $=$ Wasted Electronic and Electrical Equipment}}
		\begin{itemize}
			\item Material reasons:
			\begin{itemize}
				\item A lot of small equipments are containing very expensive (heavy/precious) metals (Au, Ag, Pb, Ta, Ni, Co, Cd, Pd, ...).
			\end{itemize}
			\item Ecological reasons:
			\begin{itemize}
				\item Using of hazardous substance during production flow
				\item Presence of some toxic substances in WEEE (batteries, accumulators, soldering, SMD components etc.)
			\end{itemize}
		\end{itemize}
		\begin{center}
			\includegraphics[width=0.65\linewidth]{obr09_elektroABaterie.png}
		\end{center}
	\end{frame}
%------------------------------------------------------------------------------
	\begin{frame}
    \frametitle{Electronic Waste Production}
		
		\begin{itemize}
			\item continual increase of no. of mobile phones, PCs,
			\item increasing complexity and sophistication of equipments.
		\end{itemize}
		\begin{center}
				\includegraphics[scale=0.55]{obr05_elektroodpad.png}
		\end{center}
	\end{frame}
%------------------------------------------------------------------------------
	\begin{frame}
    \frametitle{Impact of WEEE on environment}
		
		\begin{itemize}
			\item Consumption of non-renewable sources - use of expensive metals like Hg, Pb and a lot of not-recycled materials
			\item \textcolor{red}{Consumption of energy} - necessary for recycling of WEEE
			\item Using of hazardous chemical materials in WEEE - Cd, Ta, Co, Cr, Pd, Pb, Hg
			\item Necessity of a new places for junk
		\end{itemize}
	\end{frame}
%------------------------------------------------------------------------------
	\begin{frame}
    \frametitle{EU Attitude towards WEEE}
		
		\begin{myDef}
		The EU sets the obligation for all Member States of retrograde purchasing. Also recycling of WEEE is mandatory.
		\end{myDef}
		\textbf{Basic aims:}
		
		\begin{itemize}
			\item prevention in WEEE branch,
			\item increase of volume of recycled equipments,
			\item minimizing of WEEE as a municipal waste,
			\item responsibility is on the producer!!! (basic principle) during the whole life-time of EEE (LCA $=$ Life Cycle Assessment).

		\end{itemize}
	\end{frame}
%------------------------------------------------------------------------------
%Legislation
%------------------------------------------------------------------------------
\section{\texorpdfstring{Legislation}{Legislation}}
%------------------------------------------------------------------------------
	\begin{frame}
    \frametitle{Law no. 185/2001 Sb.}
		\small
		\textbf{Czech law about wasted materials: no. 185/2001 Sb.}
		
		\begin{itemize}
			\item It processes several EU directives about wasted materials handling.
			\begin{itemize}
				\item (\textbf{RoHS1}) 2002/95/ES Restriction of the use of certain Hazardous Substances in electrical and electronic equipment
			\end{itemize}
			\item Implementation is ensured via specific \uv{Government Regulations}.
			\item It covers responsibilities, waste sorting, fines, waste treatment etc.
		\end{itemize}
	\end{frame}
%------------------------------------------------------------------------------
	\begin{frame}
    \frametitle{Law no. 185/2001 Sb.}
		\small
		\begin{itemize}
			\item \textbf{Wasted Electronic and Electrical Equipment (\textbf{WEEE})}\\
			\begin{myDef}
			\uv{is product, which functionality depends on electrical current or on electromagnetic filed, or which is determinate for production, transport, measuring of electrical current or electromagnetic filed. It is dedicated for all equipments supplied from 1000 VAC up to 1500 VDC.}
			\end{myDef}
			\item WEEE - all household products are considered as WEEE after finishing their life-time.
		\end{itemize}
		
	\end{frame}
%------------------------------------------------------------------------------
	\begin{frame}
    \frametitle{Law no. 22/1997 Sb.}
		\small
		\textbf{Czech law about technical requirements for products: no. 22/1997 Sb.}
		\begin{itemize}
			\item It considers the requirements for devices released to the market.
			\item Several EU Directives are covered via \uv{Government Regulations}
			\begin{myDef}
			\begin{itemize}
				\item GR no. \textcolor{blue}{86/2011 Sb.} about technical requirements of toys (\textcolor{blue}{Directive 2009/48/ES}),
				\item GR no. \textcolor{blue}{54/2015 Sb.} about technical requirements of medical care equipment (\textcolor{blue}{Directives 93/42/EHS, updated in 98/79/ES, updated in 2000/70/ES, ..., updated in 2007/47/ES}),
				\item ...,
				\item \textcolor{blue}{\textbf{GR no. 481/2012 about restricted use of dangerous materials in electric devices}.}
			\end{itemize}
			\end{myDef}
		\end{itemize}
		
	\end{frame}
	
%------------------------------------------------------------------------------
	\begin{frame}
    \frametitle{Directive 2011/65/ES RoHS}
		\small
		\textbf{(\textbf{RoHS2}) 2011/65/ES Restriction of the use of certain Hazardous Substances in electrical and electronic equipment}
			
			\begin{itemize}
				\item It is covered by GR no. 481/2012 Sb. in Czech Republic
				\item Restricted materials:
				\begin{center}
				\begin{tabular}{|m{0.1\linewidth} |m{0.25\linewidth} |m{0.25\linewidth} |}
				\hline
				& Substance & Concentration\\
				\hline
				Pb & Lead & 0,1 \%\\
				Hg & Mercury & 0,1 \%\\
				Cd & Cadmium & 0,01 \%\\
				Cr & Hexavalent chromium & 0,1 \%\\
				PBB & Polybrominated biphenyls & 0,1 \%\\
				PBDE & Polybrominated diphenyl ethers & 0,1 \%\\
				\hline
				\end{tabular}
				\end{center}
			\end{itemize}
	\end{frame}
%------------------------------------------------------------------------------
	\begin{frame}
    \frametitle{Directive 2011/65/ES RoHS, Consequences}
			\begin{itemize}
				\item There is a lot of exceptions in the directive:
				
				\begin{itemize}
					\item \textbf{Hg} in fluorescent tube,
					\item \textbf{Pb} in fluorescent tube glass, as an alloying element  in steels, in some solders for specific use,
					\item \textbf{Cd} in LEDs, ...
				\end{itemize}
				\item Soldering - just Pb and Cd-free !!! (higher temperature, lower reliability, no advantages)
				\item NiCd batteries are replaced by NiMH bateries (little worse service life)
			\end{itemize}
			\begin{center}
			\includegraphics[width=0.8\linewidth]{obr10_zarivka}
		\end{center}
	\end{frame}
%------------------------------------------------------------------------------
	\begin{frame}
    \frametitle{What belongs to WEEE?}
			\begin{enumerate}
				\item Big household appliances (freezers, washing machine, cookers, etc.)
				\item Small household appliances (fan, kettle, elec. shavers)
				\item Equipment of ITC, printers, photocopiers
				\item Consumer equipment (TV, Hi-Fi, DVD player...)
				\item Lighting systems
				\item Hand-held electrical tools (screwdrivers, grinders, drills...) 
				\item Toys, sports accessories
				\item Medical equipments (exception for implanted equipments - pacemakers) 
				\item Regulating apparatus, sensors (e.g. for heating), 
				\item Automatic machines for coins, coffee, etc.
			\end{enumerate}
	\end{frame}
%------------------------------------------------------------------------------
	\begin{frame}
    \frametitle{Producer Responsibility}
			\begin{itemize}
				\item He must be registered on the Ministry of the Environment of the Czech Department
				\item He provides marking of products as follows:
				\begin{itemize}
					\item CE mark - Declaration of Conformity
					\item symbol (name) of producer
					\item symbol of WEEE
				\end{itemize}
			\end{itemize}
			\begin{center}
			\includegraphics[scale=0.4]{obr06_symbolWEEE.png}
			\end{center}
	\end{frame}
%------------------------------------------------------------------------------
	\begin{frame}
    \frametitle{Waste Management}
		\begin{myDef}
		\begin{itemize}
			\item Producers must provide collection point for WEEE,
			\item customers must be informed about these places.
		\end{itemize}
		\end{myDef}
		Obligations:
		
		\begin{itemize}
			\item Reverse collection
			\begin{itemize}
				\item Collecting of used WEEE from household.
				\item Collecting of used WEEE from companies, from production lines, etc.
			\end{itemize}
			\item Usage of WEEE
			\begin{itemize}
				\item Using once again without any other processing - good idea, but it does not work at all.
				\item Separation of basic materials and elements for second usage (energy and processing necessary).
			\end{itemize}
		\end{itemize}
	\end{frame}
%------------------------------------------------------------------------------
	\begin{frame}
    \frametitle{Companies Recycling WEEE}
		\begin{center}
		\begin{tabular}{m{0.4\linewidth} m{0.5\linewidth}}
		\includegraphics[scale=0.4]{obr07_popelniceElektro.png} &
		
		\begin{itemize}
			\item ASEKOL
			\item REMA
			\item RETELA
			\item SAFINA a.s.
			\item Kovohutě Příbram a.s.
			\item MHM eko s.r.o.
		\end{itemize}
		\end{tabular}
		\includegraphics[scale=0.4]{obr08_popelniceRuzne.png}
		\end{center}
	\end{frame}
%------------------------------------------------------------------------------
\end{document}